\documentclass[english,11pt,a4paper]{article}

\usepackage{pgfplots}
\pgfplotsset{compat=newest}
%% the following commands are sometimes needed
\usetikzlibrary{plotmarks}
\usepackage{grffile}

\usepackage{latexsym}
\usepackage{babel}
% %\usepackage{rotating}
\usepackage{subfig}
%\usepackage{multibbl}
%\usepackage{calc}
%\usepackage{lastpage}
\usepackage{amsmath,epsfig,amssymb,amsbsy, amsthm}
%\usepackage{euproposal}
\usepackage{multirow}
%\usepackage{natbib}
%\usepackage{multibib}
%\usepackage{fancybox}
\usepackage{nicefrac}
%\usepackage[table]{xcolor}
%\usepackage[right]{eurosym}



\usepackage[bordercolor=white,backgroundcolor=gray!30,linecolor=black,colorinlistoftodos]{todonotes}

%\usepackage[T1]{fontenc}

\usepackage{wrapfig,graphicx}
\usepackage{url}
%\usepackage{titlesec}

\usepackage{color}
\usepackage{colortbl}

\newcommand{\todobase}[3]{\todo[color=#3,inline]{#2}}

\newcommand{\jl}[1]{\todobase{J}{#1}{yellow}}
\newcommand{\tmo}[1]{\todobase{J}{#1}{pink}}
\newcommand{\mm}[1]{\todobase{J}{#1}{cyan}}

\renewcommand{\baselinestretch}{1.2} 

\newcommand\dimOmega{d}
\newcommand{\ngradients}{m}
\newcommand{\mfindex}{k} %k
\newcommand{\unarycost}{s}
\newcommand{\mfweights}{b}
\newcommand{\bitem}{\begin{itemize}}
\newcommand{\eitem}{\end{itemize}}
\newcommand{\mc}[1]{\mathcal{#1}}
\newcommand{\mb}[1]{\mathbb{#1}}
\newcommand{\R}{\mathbb{R}}
\newcommand{\B}{\mathbb{B}}
\newcommand{\U}{\mathbb{U}}
\newcommand{\E}{\mathbb{E}}
\newcommand{\Q}{\mathbb{Q}}
\newcommand{\Z}{\mathbb{Z}}
\newcommand{\PP}{\mathbb{P}}
\newcommand{\bpm}{\begin{pmatrix}}      
\newcommand{\epm}{\end{pmatrix}}
\newcommand{\T}{\top}
\newcommand{\tr}{\mathsf{tr}}
\newcommand{\ol}[1]{\overline{#1}}      
\newcommand{\la}{\langle}
\newcommand{\ra}{\rangle}
\newcommand{\si}{\sigma}
\newcommand{\mrm}[1]{\mathrm{#1}}
\newcommand{\msf}[1]{\mathsf{#1}}
\newcommand{\Diag}{\mathrsf{Diag}}
\newcommand{\row}[2]{{#1}_{#2,\bullet}}
\newcommand{\col}[2]{{#1}_{\bullet,#2}}
\newcommand{\veps}{\varepsilon}
\newcommand{\toset}{\rightrightarrows}
\newcommand{\w}{\omega}
\newcommand{\gdw}{\Leftrightarrow}
\newcommand{\vphi}{\varphi}
\newcommand{\tmop}[1]{\ensuremath{\operatorname{#1}}}
\newcommand{\norm}[1]{\Vert #1 \Vert}
\newcommand{\normc}[1]{| #1 |}
\newcommand{\bbR}{\mathbb{R}}
\newcommand{\bbRext}{\mathbb{R} \cup \{ \infty \} }
\newcommand{\bbN}{\mathbb{N}}
\providecommand{\iprod}[2]{\langle#1,#2\rangle}


\newcommand{\bi}{\begin{itemize}}
\newcommand{\ei}{\end{itemize}}
\newcommand{\HH}{\mathcal{H}}
\newcommand{\MM}{\mathcal{M}}

\newcommand{\DD}{\mathcal{D}}
\newcommand{\JJ}{\mathcal{J}}
\newcommand{\Ss}{\mathcal{S}}

\newcommand{\reps}{\rho_{\veps}}
\newcommand{\geqs}{\geqslant}
\newcommand{\leqs}{\leqslant}


%\newcommand{\cref}[1]{ [{#1}]}
\newcommand{\cref}[1]{ {\tiny[{#1}]}}
%\newcommand{\tm}[1]{[#1] }
\newcommand{\tm}[1]{}

\newcommand{\VI}{\mathsf{VI}}
\newcommand{\SOL}{\mathsf{SOL}}
\newcommand{\graph}{\mathsf{gph}}
\newcommand{\dom}{\mathsf{dom}}
\newcommand{\conv}{\mathsf{conv}}
\newcommand{\rint}{\mathsf{rint}}
\newcommand{\bd}{\mathsf{bd}}
\newcommand{\rge}{\mathsf{rge}}
\newcommand{\epi}{\mathsf{epi}}
\newcommand{\lev}{\mathsf{lev}}
\newcommand{\argmin}{\mathsf{argmin}}
\newcommand{\argmax}{\mathsf{argmax}}

\newcommand{\detail}{\ding{43} \structure{Detail}}

\newcommand{\fr}[3][]{\frame{\frametitle{#2}\begin{textblock*}{6cm}(12cm,1.5cm)\rotatebox{90}{\parbox{7cm}{\scriptsize\structure{#1}}}\end{textblock*}\bi #3 \ei}}
\newcommand{\frb}[3][]{\frame{\frametitle{#2}\begin{textblock*}{6cm}(12cm,1.5cm)\rotatebox{90}{\parbox{7cm}{\scriptsize\structure{#1}}}\end{textblock*}#3}}
\newcommand{\tmemtitle}{\textbf}

\newcommand{\mathbbm}{\mathbb}
\newcommand{\C}{\mathcal{C}}
\newcommand{\D}{\mathcal{D}}
\newcommand{\Rn}{\mathbb{R}^n}
\newcommand{\Rm}{\mathbb{R}^m}
\newcommand{\Rmxn}{\mathbb{R}^{m \times n}}
\newcommand{\Rd}{\mathbb{R}^d}
\newcommand{\Rl}{\mathbb{R}^L}
\newcommand{\beq}{\begin{equation}}
\newcommand{\eeq}{\end{equation}}
\newcommand{\beqa}{\begin{eqnarray}}
\newcommand{\eeqa}{\end{eqnarray}}
\newcommand{\bc}{\begin{center}}
\newcommand{\ec}{\end{center}}
\newcommand{\lcb}{\left\{}
\newcommand{\rcb}{\right\}}
\newcommand{\lbr}{\left(}
\newcommand{\rbr}{\right)}
\newcommand{\seq}{\subseteq}
\newcommand{\PC}{\Pi_{\C}}
\newcommand{\PD}{\Pi_{\D}}
\newcommand{\assign}{:=}
\newcommand{\Om}{\Omega}
\newcommand{\BV}{\tmop{BV}}
\newcommand{\SBV}{\tmop{SBV}}
\newcommand{\Ccinf}{C_c^{\infty}}
\newcommand{\Dloc}{\D_{\tmop{loc}}}
\newcommand{\Hausd}{\mathcal{H}}
\newcommand{\tmtextbf}[1]{{\bfseries{#1}}}
\newcommand{\tmtextit}[1]{{\itshape{#1}}}
\newcommand{\tmem}[1]{{\em #1\/}}
%\newcommand{\subs}[1]{\smallskip\noindent\textbf{#1.} }
\newcommand{\subs}[1]{\medskip\noindent\textbf{#1.} }
%\newcommand{\subs}[1]{\noindent\textbf{#1.} }
%\newcommand{\subs}[1]{\paragraph{#1.} }

\newcommand{\wi}{.1\columnwidth}

\newcommand \M   {{\mathcal{M}}}                     % manifold

\newcommand \Sone       {{{\Ss}^1}}                         % S^1
\newcommand \TVS        {{TV_\Sone}}                        % TV_{S^1}
\newcommand \TVM        {{TV_\M}}                           % TV_{S^1}
\newcommand \Stwo       {{{\Ss}^2}}                         % S^1
\newcommand \TVStwo     {{TV_\Stwo}}                        % TV_{S^1}
\newcommand \TVSg       {{TV_\Sone^g}}                      % TV_{S^1}^g
\newcommand \TV         {{TV}}                              % TV
\newcommand{\dx}        {\,dx}
\newcommand \Su         {{S_u}}                             % subset
% of Omega w 
\newcommand \omegaDim   {{m}}                               % image domain dimension
\newcommand \dH         {{d\;\!\mathcal{H}^{\omegaDim-1}}}  % Hausdorff measure
\newcommand{\Div}{\tmop{Div}}


%\arrayrulecolor{black} % green tables otherwise
%\renewcommand{\familydefault}{\sfdefault}
%\usepackage{arial}

\newcommand{\fc}{^{\ast}}
\newcommand{\fcc}{^{\ast\ast}}
\newcommand{\ow}{\text{otherwise}}

\newcommand{\RL}{\R^L}
\newcommand{\Rb}{\bar{\R}}

%\newtheorem{thm}{Theorem}[section]
%\newtheorem{prop}[thm]{Proposition}

\newtheorem{lem}{Lemma}
\newtheorem{prop}{Proposition}
\newtheorem{cor}{Corollary}
\newtheorem{rem}{Remark}

\newcommand{\dat}{\boldsymbol{\rho}}
\newcommand{\dats}{\boldsymbol{\sigma}}
\newcommand{\reg}{\boldsymbol{\Phi}}
\newcommand{\Gr}{\mathbf{1}}
\newcommand{\ul}{\boldsymbol{u}}
\newcommand{\el}{\boldsymbol{e}}
\newcommand{\vl}{\boldsymbol{v}}
\newcommand{\Vl}{\boldsymbol{V}}
\newcommand{\gl}{\boldsymbol{g}}
\newcommand{\pl}{\boldsymbol{p}}
\newcommand{\ql}{\boldsymbol{q}}
\newcommand*\Bell{\ensuremath{\boldsymbol\ell}}
\newcommand*\Bgamma{\ensuremath{\boldsymbol\gamma}}

\newcommand*{\vertbar}{\rule[-1ex]{0.5pt}{2.5ex}}

\newcommand\inner[2]{\langle #1, #2 \rangle}

\usepackage{a4wide}
\usepackage{tikz}

\title{Ideas for Splitting Deep Networks}
\author{}
\usepackage{amsmath,epsfig,amssymb,amsbsy}

\begin{document}
\maketitle

\section{Objective function}

The aim is to solve a supervised regression or classification problem involving a highly nested objective function. A mapping $f(x)$ from the inputs $x$ to the corresponding outputs $y$ based on $N$ training samples $(x_n, y_n)$ should be learned. The loss function can be formulated as

\begin{equation}
E(W) = \frac{1}{2} \sum_{i=1}^{N} L(y_i, f(W;x_i))
\end{equation}

with some appropriate error measure $L(y,x)$. This error measure can for example be the squared l2-loss $L(y,x) = \frac{1}{2} \| y - x \|^2_2$. The function $f(W;x) = W_{K+1}h(W_Kh(\dots h(W_1x))$ is a nested K-layer mapping as it is usual in neural networks where $W$ are the weight matrices $W_1,\dots,W_{K+1}$ which do not need to have the same size. The nonlinear activation function $h(x)$ is applied element-wise and can also be non-differential. Examples are the sigmoid function $h(x) = 1/(1 + e^{-x})$ or the ReLU $h(x) = \mathrm{max}(0, x)$.

\section{Solution concept}

In the following, we stack all input samples into the matrix $x$ where each column is one sample point. Similarly, the target variables are stacked into the matrix $y$. The error function $L$ now operates on matrices. This formulation is equal to the formulation above, however, the layers now operate on all the data simultaneously. Additionally, without loss of generality we consider a 2-layer neural network mapping $f(W;x) = W_3h(W_2h(W_1x))$ with a linear mapping to the output variables. The unconstrained optimization problem can then be written as

\begin{equation}
	\begin{aligned}
		& \underset{\{W_i\}}{\text{minimize}}
		& & L(y,W_3h(W_2h(W_1x)).
	\end{aligned}
\end{equation}

To split the nested objective function we decouple the weights and activation functions
by introducing additional variables and adding the appropriate constraints. This reformulated problem is equal to the original problem and can be written as

\begin{equation}
	\begin{aligned}
		& \underset{\{W_i,a_i,z\}}{\text{minimize}}
		&& L(y,z). \\
		& \text{subject to}
		&&  a_1 = h(W_1x), \\
		&&& a_2 = h(W_2a_1), \\
		&&& z = W_3a_2
	\end{aligned}
\end{equation}

with the new variables $a_1, a_2, z$. The augmented Lagrangian of this equally constrained problem is

\begin{equation}
	\begin{aligned}
		\mathcal{L}(y,z,a,\lambda) = L(y,z) + 
		& \inner{\lambda_1}{h(W_1x)-a_1} + \rho/2 \| h(W_1x)-a_1 \|^2 + \\
		& \inner{\lambda_2}{h(W_2a_1)-a_2} + \rho/2 \| h(W_2a_1)-a_2 \|^2 + \\
		& \inner{\lambda_3}{W_3a_2-z} + \rho/2 \| W_3a_2-z \|^2.
	\end{aligned}
\end{equation}

with $a = (a_1,a_2)^T$, $\lambda = (\lambda_1, \lambda_2, \lambda_3)^T$ and $\rho \in \R$. For minimizing the objective we update the weights and constraint variables of subproblems in an alternating fashion using gradient descent. In each update step we only consider the variables of one constraint and fix all other terms of the augmented Lagrangian. This results in one subproblem $\mathcal{H}_i$ for each constraint.

\begin{equation}
	\begin{aligned}
		\mathcal{H}_1(W_1,a_1,\lambda_1) &= && \inner{\lambda_1}{h(W_1x)-a_1} + \rho/2 \| h(W_1x)-a_1 \|^2 + \\
			& && \inner{\lambda_2}{h(W_2a_1)-a_2} + \rho/2 \| h(W_2a_1)-a_2 \|^2 \\
		\mathcal{H}_2(W_2,a_1,a_2,\lambda_2) &= && \inner{\lambda_2}{h(W_2a_1)-a_2} + \rho/2 \| h(W_2a_1)-a_2 \|^2 + \\
			& && \inner{\lambda_1}{h(W_1x)-a_1} + \rho/2 \| h(W_1x)-a_1 \|^2 + \\
			& && \inner{\lambda_3}{W_3a_2-z} + \rho/2 \| W_3a_2-z \|^2 \\
		\mathcal{H}_3(W_3,a_2,z,\lambda_3) &= && \inner{\lambda_3}{W_3a_2-z} + \rho/2 \| W_3a_2-z \|^2 + L(y,z) \\
			& && \inner{\lambda_2}{h(W_2a_1)-a_2} + \rho/2 \| h(W_2a_1)-a_2 \|^2 
	\end{aligned}
\end{equation}

We need to take special care of the first subproblem (constraint for the first layer) involving the input $x$ and the last subproblem (constraint for the last linear layer) involving the predicted variable $z$. For all others we consider the update scheme (here for $\mathcal{H}_2$)

\begin{equation}
	\begin{aligned}
		W_2^{t+1} &= W_2^t - \rho_1 \nabla_{W_2}(\mathcal{H}_2(W_2^t,a_1^t,a_2^t,\lambda_2^t)) \\
		a_1^{t+1} &= a_1^t - \rho_2 \nabla_{a_1}(\mathcal{H}_2(W_2^t,a_1^t,a_2^t,\lambda_2^t)) \\
		a_2^{t+1} &= a_2^t - \rho_3 \nabla_{a_2}(\mathcal{H}_2(W_2^{t+1},a_1^{t+1},a_2^t,\lambda_2^t)) \\
		\lambda_2^{t+1} &= \lambda_2^{t} + \rho (h(W_2^{t+1}a_1^{t+1})-a_2^{t+1})
	\end{aligned}
\end{equation}

which are gradient descent steps for the primal variables and gradient ascent steps for the dual variable $\lambda$. For $\mathcal{H}_1$ only $W_1$, $a_2$ and $\lambda_1$ need to be updated since $x$ is a constant. For the last subproblem $\mathcal{H}_{K+1}$ we need to add $\nabla_zL(y,z)$ to the gradient with respect to $z$.

\section{Appendix}

\subsection{Notation}

For a matrix $X$, $X_{ij}$ selects the element in row $i$ and column $j$. If only one index is present, depending on the subscript a matrix row or column is select. In this case, $X_i$ is row $i$, $X_j$ is column $j$.

\subsection{Energy gradients}

For the update steps we need the gradient of the terms $f(W,x,a,\lambda) = \inner{\lambda}{h(Wx)-a}$ and $g(W,x,a) = (1/2) \| h(Wx) - a \|^2$ with respect to $W \in \R^{m \times n}$, $x \in \R^{n \times p}$, $a \in \R^{m \times p}$ and $\lambda \in \R^{m \times p}$. The function $h$ is applied element-wise. Note that for the inner product the matrices are vectorized. \\ \\
For the gradients we need the following intermediate results. For applying the chain-rule the derivatives of $h(Wx) = h \in \R^{m \times p}$ w.r.t $W$ and $x$ are required. Taking the derivative for each element the entries for the Jacobian are $\partial_{W_i}h_{ij} = h'(W_i x_j) x_j^T$. Similarly, for the derivatives w.r.t. $x$ we get $\partial_{x_j}h_{ij} = h'(W_i x_j) W_i^T $. Note that we can easily get a formulation for all matrix elements at once by the fact that for the Jacobians it holds that $J_{(f \circ g)} = J_f \cdot J_g$. This results in

\begin{equation}
	\begin{aligned}
		\partial_{W_i} f(W,x,a,\lambda) &= \sum_{j=1}^{p} \lambda_{ij} h'(W_i x_j) x_j^T \\
		\partial_{x_j} f(W,x,a,\lambda) &= \sum_{i=1}^{m} \lambda_{ij} h'(W_i x_j) W_i^T \\
		\partial_{\lambda} f(W,x,a,\lambda) &= h(W x) - a \\
		\partial_{a} f(W,x,a,\lambda) &= -\lambda
	\end{aligned}
\end{equation}

for the gradients of $f$. For $g$ the gradients are

\begin{equation}
	\begin{aligned}
		\partial_{W_i} g(W,x,a) &= \sum_{j=1}^{p} (h(W_i x_j) - a_{ij}) h'(W_i x_j) x_j^T \\
		\partial_{x_j} g(W,x,a) &=  \\
		
		\partial_{a} g(W,x,a) &= -h(Wx) + a \\
		\partial_{x} g(W,x,a) &= (h(Wx)-a) W^Th'(Wx).
	\end{aligned}
\end{equation}

Note that the expressions are in a format where we get a short formula. For calculating the matrix derivatives for all rows or for the complete matrix at once some matrix stacking is required which would result in a longer expression.

\section{Additional notes}

\begin{itemize}
	\item Could add regularizer for the weights.
\end{itemize}

{\small
  \bibliographystyle{ieee}
  \bibliography{references}
}

\end{document}

